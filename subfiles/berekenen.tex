\documentclass[../main.tex]{subfiles}
\begin{document}

\subsection{Vraag 1}
\begin{question}
	In de loop van de geschiedenis hebben wetenschappers/ingenieurs in verschillende
	contexten gedacht dat alles geconstrueerd/berekend kon worden, en dikwijls werd 
	dat ook (later) tegengesproken. Geef daarvan een aantal voorbeelden en geef hun 
	historische verbanden in het bijzonder voor de ontwikkelingen i.v.m. functies/algoritmen 
	tijdens de 20ste eeuw. Bespreek ook in hoeverre die inzichten gestuurd zijn door 
	technologische vernieuwingen.
\end{question}

\begin{solution}
	
\end{solution}


\subsection{Vraag 2}
\begin{question}
	Schets de oorsprong en de evolutie van computationele complexiteitsleer in de 20ste
	eeuw.
\end{question}

\begin{solution}
	
\end{solution}


\end{document}