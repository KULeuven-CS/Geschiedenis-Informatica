\documentclass[../main.tex]{subfiles}
\begin{document}

\subsection{Vraag 1}
\begin{question}
Geef en bespreek enkele belangrijke mijlpalen in de geschiedenis van het internet.
\end{question}

\begin{solution}
Algemene mijlpalen:
\begin{itemize}
	\item 1961: Eerste onderzoek met ondermeer paper over \textbf{packet switching} door Leonard Kleinrock (MIT).
	\item 1969: Eerste twee knopen van ARPANET: verbinding tussen UCLA en SRI. Al snel wordt het netwerk uitgebreid tot 4 nodes met UCSB en University of Utah.
	\item 1972: Eerste e-mail over ARPANET en eerste publieke demonstratie.
	\item 1974: Samenwerking \textbf{Vint Cerf} en \textbf{Robert Kahn} leidt tot TCP/IP protocol dat NCP vervangt. Vier jaar later in 1978 wordt TCP en IP gesplitst wat het ontstaan van UDP toelaat. In 1982 wordt TCP de als standaard gedeclareerd door de DoD.
	\item 1984: Komst van \textbf{DNS}. Het jaar daarop wordt USC verantwoordelijk voor DNS root beheer en SRI (Stanford) voor NIC registraties.
	\item 1985: Oprichting van het \textbf{NSFnet} (National Science Foundation Network), research netwerk in de Verenigde Staten.
	\item 1989: Definitie WWW door Tim Berners-Lee bij CERN.
\end{itemize}
Mijlpalen specifiek voor Belgi\"e
\begin{itemize}
	\item 1988: Ontstaan \textbf{.be} TLD.
	\item 1991: Verbinding met het Internet.
	\item 1994: Start commercieel internetdiensten.
\end{itemize}
\end{solution}


\subsection{Vraag 2}
\begin{question}
Velen stellen dat het beheer van het internet doorheen de jaren nogal chaotisch is
verlopen. Geef uw visie daarop.
\end{question}

\begin{solution}

\end{solution}



\subsection{Vraag 2}
\begin{question}
Geef enkele belangrijke stappen bij de invoering van het internet in België.
\end{question}

\begin{solution}

\end{solution}


\end{document}