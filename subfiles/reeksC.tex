\documentclass[../main.tex]{subfiles}
\begin{document}

\subsection{Reeks C}
\subsubsection{Vraag 1}
\begin{question}
Bespreek het onderscheid tussen mainframe, minicomputer, personal computer, werkstation, supercomputer.
\end{question}

\subsubsection{Vraag 2}
\begin{question}
Toon met voorbeelden aan dat belangrijke internationale gebeurtenissen een grote rol hebben gespeeld in de ontwikkeling van computers en informatica.
\end{question}

\subsubsection{Vraag 3}
\begin{question}
IBM heeft een belangrijke rol gespeeld in de ontwikkeling van computers en informatica. Kan je belangrijke verwezenlijkingen of mijlpalen aangeven?
\end{question}

\subsubsection{Vraag 4}
\begin{question}
60 jaar geleden: 1954. Kan je je voorstellen wat er toen nog niet was (in het dagelijkse leven), en dat gerealiseerd is door middel van computers, processoren, informatica?
\end{question}

\subsubsection{Vraag 5}
\begin{question}
Kan je een belangrijke gebeurtenis of evolutie op het gebied van de informatica voor elk  decennium vanaf 1940-1949 noemen en toelichten?
\end{question}

\subsubsection{Vraag 6}
\begin{question}
Hoe zou je zelf de geschiedenis van de informatica in periodes indelen? Op grond waarvan?
\end{question}

\subsubsection{Vraag 7}
\begin{question}
Wat weet je over de geschiedenis en de voorlopers van het WWW?
\end{question}

\subsubsection{Vraag 8}
\begin{question}
Bespreek kort de evolutie van besturingssystemen en gebruikersinterfaces.
\end{question}

\subsubsection{Vraag 9}
\begin{question}
Hoe zijn programmeren en software engineering geëvolueerd?
\end{question}

\subsubsection{Vraag 10}
\begin{question}
Schets de evolutie van de verschillende soorten programmeertalen aan de hand van het bijgevoegd (vereenvoudigd) overzicht. (Zie foto's voor overzicht)
\end{question}
\end{document}