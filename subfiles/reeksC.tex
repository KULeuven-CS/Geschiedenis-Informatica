\documentclass[../main.tex]{subfiles}
\begin{document}

\subsection{Reeks C}
\subsubsection{Type Computers}
\begin{question}
Bespreek het onderscheid tussen mainframe, minicomputer, personal computer, werkstation, supercomputer.
\end{question}
\begin{solution}
\end{solution}

\subsubsection{Invloed Internationale Gebeurtenissen}
\begin{question}
Toon met voorbeelden aan dat belangrijke internationale gebeurtenissen een grote rol hebben gespeeld in de ontwikkeling van computers en informatica.
\end{question}
\begin{solution}
		We vinden hiervan verschillende voorbeelden terug doorheen de geschiedenis.
		Reeds eind negentiende eeuw leid de grote migratie naar amerika tot het probleem van volkstellingen.
		Dit gaf de aanzet tot het ontnwikkelen van de Hollerith Machine met als doel het vereenvoudigen van de volkstelling.

		Een volgende heel duidelijk voorbeeld is tijdens de tweede wereld oorlog.
		Vooral in Amerika en Groot Brittanni\"e is de invloed zeer duidelijk te merken. 
		De Collosus I (GB) wordt in het geheim ontwikkeld om duitse geheime codes te breken. 
		Ook de ASCC, ENIAC (VS) van zijn opvolgers werden gebruikt door het Amerikaanse leger om o.a. projectielbanen te berekenen. 
		In Duitsland was er door de oorlog vooral een negatieve invloed te merken.
		Zo kreeg Konrad Zuse door de oorlogskosten geen subsidie voor het ontwikkelen van zijn Z-reeks.
		Ook na de oorlog was het door gebrek aan geld en middelen relatief stil op het front van computers in duitsland.	

		De ontwikkeling van het ARPAnet (voorloper van het internet) is ook gedeeltelijk toe te schrijven aan de koude oorlog. 
		Er was immers nood aan communicatie als er verschillende nodes in een netwerk zouden uitvallen.
\end{solution}

\subsubsection{IBM}
\begin{question}
IBM heeft een belangrijke rol gespeeld in de ontwikkeling van computers en informatica. Kan je belangrijke verwezenlijkingen of mijlpalen aangeven?
\end{question}

\subsubsection{Computerrealisaties}
\begin{question}
60 jaar geleden: 1954. Kan je je voorstellen wat er toen nog niet was (in het dagelijkse leven), en dat gerealiseerd is door middel van computers, processoren, informatica?
\end{question}

\subsubsection{Gebeurtenis Decenia}
\begin{question}
Kan je een belangrijke gebeurtenis of evolutie op het gebied van de informatica voor elk decennium vanaf 1940-1949 noemen en toelichten?
\end{question}

\subsubsection{Periodes}
\begin{question}
Hoe zou je zelf de geschiedenis van de informatica in periodes indelen? Op grond waarvan?
\end{question}

\subsubsection{World Wide Web}
\begin{question}
Wat weet je over de geschiedenis en de voorlopers van het WWW?
\end{question}

\subsubsection{Besturingssystemen \& Gebruikersinterfaces}
\begin{question}
Bespreek kort de evolutie van besturingssystemen en gebruikersinterfaces.
\end{question}

\subsubsection{Programmeren \& Software Engineering}
\begin{question}
Hoe zijn programmeren en software engineering geëvolueerd?
\end{question}

\subsubsection{Evolutie Programmeertalen}
\begin{question}
Schets de evolutie van de verschillende soorten programmeertalen aan de hand van het bijgevoegd (vereenvoudigd) overzicht. (Zie foto's voor overzicht)
\end{question}
\end{document}
